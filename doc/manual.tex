% This file is the user manual for the UGKS1D and UGKS2D code
%
% Copyright (C) 2012 Wang Ruijie <lainme993@gmail.com> 
%
% Permission is granted to copy, distribute and/or modify this document
% under the terms of the GNU Free Documentation License, Version 1.3 or
% any later version published by the Free Software Foundation; with no
% Invariant Sections, with no Front-Cover Texts, and with no Back-Cover
% Texts.  A copy of the license is included in the section entitled
% ``GNU Free Documentation License.''

\documentclass[a4paper]{book}
\usepackage{hyperref}
\usepackage{parskip}
\usepackage{amsmath}
\usepackage{fullpage}
%no indention
\setlength\parindent{0pt}

\begin{document}
%title page
\title{User Manual for UGKS1D and UGKS2D Code}
\author{Wang Ruijie\\\href{mailto:lainme993@gmail.com}{lainme993@gmail.com}}
\maketitle

%license statement
\thispagestyle{empty}
Copyright \copyright\ 2012 Wang Ruijie

Permission is granted to copy, distribute and/or modify this document
under the terms of the GNU Free Documentation License, Version 1.3 or
any later version published by the Free Software Foundation; with no
Invariant Sections, no Front-Cover Texts, and no Back-Cover Texts.  A
copy of the license is included in the section entitled
\htmlref{GNU Free Documentation License}{chap:fdl}

%table of contents
\frontmatter
\tableofcontents

\mainmatter

%UGKS method
\chapter{Unified Gas-Kinetic Scheme}
This chapter describes the Unified Gas-Kinetic Scheme presented in \cite{Xu2010,Xu2011}. This is a 1D formulation. The 2D formulation with directional splitting is presented in \cite{Huang2012}, and can be extended to a truely multidimensional formulation, see\cite{Xu2005}.

\section{Model equation}
The model equation is the BGK-Shakhov model. In one dimensioanl case,
\begin{equation}
    \label{eq:bgk-shakhov}
    \frac{\partial f}{\partial t}+u\frac{\partial f}{\partial x}=\frac{f^+-f}{\tau}
\end{equation}
where $f$ is the distribution function, $u$ is partical velocity, $\tau$ is particle collision time and $f^+$ is the modified equilibrium distribution function.

The collision time is given by $\tau=\mu/p$, where $\mu$ is the dynamic viscosity and $p$ is the pressure

The modified equilibrium distribution is
\begin{equation}
    f^+ = g\left[1+(1-\Pr){\bf c}\cdot{\bf q}\left(\frac{c^2}{RT}-5\right)/(5pRT)\right]=g+g^+
\end{equation}
where $g$ is the Maxwellian distribution, $\Pr$ is the Prandtl number, ${\bf c}$ is the random velocity, $q$ is heat flux, $R$ is gas constant and $T$ is the temperature.

The Maxwellian distribution for 1D problem is
\begin{equation}
    g = \rho\left(\frac{\lambda}{\pi}\right)^{\frac{K+1}{2}}e^{-\lambda((u-U)^2+\xi^2)}
\end{equation}
where $\rho$ is density, $\lambda=m/2kT$, $m$ is molecule mass, $k$ is Boltzmann constant, $K$ is the number of internal degree of freedom and $\xi^2=\xi_1^2+\xi_2^2...+\xi_K^2$. For example, a monatomic gas at 1D problem has $K=2$ to account for the motion in $y,z$ direction, and $\xi^2=v^2+w^2$, where $v,w$ are particle velocity in $y,z$ direction.

The collision invariants are $\psi=(1,u,1/2(u^2+\xi^2))^T$, and the macroscopic variables can be calculated via
\begin{equation}
w=\begin{pmatrix} \rho\\ \rho U\\ \rho E \end{pmatrix} = \int \psi fd\Xi
\end{equation}
\begin{equation}
    p=\frac{1}{3}\int [(u-U)^2+\xi^2]fd\Xi
\end{equation}
\begin{equation}
    q=\frac{1}{2}\int (u-U)[(u-U)^2+\xi^2]fd\Xi
\end{equation}
where $U$ is the macroscopic velocity, $E$ is total energy and $d\Xi=dud\xi$

An integral solution of the BGK-Shakhov model can be constructed by the method of characteristics\cite{Prendergast1993},
\begin{equation}
    \label{eq:csolution}
    f(x,t,u,\xi)=\frac{1}{\tau}\int_{t^n}^t f^+(x',t',u,\xi)e^{-(t-t')/\tau}dt'+e^{-(t-t^n)/\tau}f_0^n(x-u(t-t^n),t^n,u,\xi)
\end{equation}
where $x'=x-u(t-t')$ is the particle trajectory and $f_0^n$ is the initial gas distribution function at $t^n$

\section{Solution algrithm}
In finite volume approach, if trapezoidal rule is used for the approximiation of collision term, Eq. \ref{eq:bgk-shakhov} becomes,
\begin{equation} 
    \label{eq:bgk-shakhov_discrete}
    f_i^{n+1} = f_i^n+\frac{1}{\Delta x}\int_{t^n}^{t^{n+1}}({\bf F}_{i-1/2}-{\bf F}_{i+1/2})dt+\frac{\Delta t}{2}\left(\frac{f^{+(n+1)}-f^{n+1}}{\tau^{n+1}}+\frac{f^{+(n)}-f^n}{\tau^n}\right)
\end{equation}
where $f_i^n$ and $f_i^{n+1}$ are cell averaged distribution function of the i-th cell at time $t=t^n$ and $t=t^{n+1}$ respectively, $\Delta x$ is the cell length and $\Delta t$ is the time step, ${\bf F}_{i-1/2}$ and ${\bf F}_{i+1/2}$ are the flux of the distribution function across the interface, $f^{+(n)}$ and $f^{+(n+1)}$ are modified equilibrium distribution, $\tau^n$ and $\tau^{n+1}$ are particle collision time.

Multiplying the collision invarients to Eq. \ref{eq:bgk-shakhov_discrete} and make integration over the velocity space, the evolution of conservative variables becomes
\begin{equation} 
    \label{eq:conservative_discrete}
    w^{n+1} = w^n+\frac{1}{\Delta x}\int_{t^n}^{t^{n+1}}\int\psi({\bf F}_{i-1/2}-{\bf F}_{i+1/2}){\rm d\Xi dt}
\end{equation}

In order to update the distribution function in Eq. \ref{eq:bgk-shakhov_discrete}, there are three unknowns should be obtained: the flux ${\bf F}$, the modified equilibrium distribution $f^{+(n+1)}$ and collision time $\tau^{n+1}$ at the next time level.

The flux ${\bf F}$ is calculated by using the integral solution Eq. \ref{eq:csolution} at the cell interface. Since $f^{+(n+1)}$ and $\tau^{n+1}$ have one-to-one correspondance to the macroscopic variables, they can be obtained by updating the conservative variables first using Eq. \ref{eq:conservative_discrete}.

In order to remove the dependence of distribution functions on the internal degree of freedom $\xi$, the reduced distribution function \cite{Yang1995} is used in real computation, which is defined as
\begin{equation} 
    h = \int_{-\infty}^{\infty}fd\xi,\quad b = \int_{-\infty}^{\infty}\xi^2 f d\xi
\end{equation} 
And the corresponding reduced Maxwellian distribution $g$ becomes
\begin{equation} 
    H = \int_{-\infty}^{\infty}gd\xi = \rho\left(\frac{\lambda}{\pi}\right)^{1/2}e^{-\lambda(u-U)^2},\quad B = \int_{-\infty}^{\infty}\xi^2 gd\xi =\frac{k}{2\lambda}H
\end{equation}

The corresponding reduced $g^+$ becomes
\begin{equation}
    \begin{aligned}
        & H^+ = \int_{-\infty}^{\infty}g^+d\xi = \frac{4(1-\Pr)\lambda^2}{5\rho}(u-U)q(2\lambda(u-U)^2+k)H \\
        & B^+ = \int_{-\infty}^{\infty}\xi^2g^+d\xi = \frac{4(1-\Pr)\lambda^2}{5\rho}(u-U)q(2\lambda(u-U)^2+k+2)B
    \end{aligned}
\end{equation}

Then the update of $f$ using Eq. \ref{eq:bgk-shakhov_discrete} becomes two similar equations for the update of $h$ and $b$, respectively

The overview flow chart of the solution algrithm is as follows
\section{Nondimensionalization}
\section{Time step and reconstruction}
\section{Calculation of interface flux}
\section{Update cell averaged value}
\section{Boundary condition}

%usage
\chapter{Usage}
This chapter describes how to compile the source code and documentation.
Currently, it's only tested under Linux.
\section{Compile under *nix}
\section{Compile under windows}

%UGKS1D code
\chapter{UGKS1D Code}
This chapter describes the structure and the included shock structure test case at Ma=8.0 in the UGKS1D code.

%UGKS2D code
\chapter{UGKS2D Code}
This chapter describes the structure and the included lid-driven cavity test case in the UGKS2D code.
\section{test}
test

%license
\backmatter
\input{fdl}

%References
\cleardoublepage
\phantomsection
\addcontentsline{toc}{chapter}{Bibliography}
\bibliographystyle{unsrt}
\bibliography{manual}
\end{document}
